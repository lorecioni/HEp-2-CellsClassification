\documentclass[a4paper,12pt]{article} 
\usepackage[utf8]{inputenc} 
\usepackage[italian]{babel} 
\usepackage{enumerate}
\usepackage{enumitem}
\usepackage{graphicx}
\usepackage{listings}
\usepackage{wrapfig}
\usepackage{subfigure}
\usepackage[a4paper,top=3cm,bottom=3cm,left=3cm,right=3cm]{geometry}
\usepackage{array}
\usepackage{hyperref}
\usepackage{tablefootnote}
\usepackage{float}
\setlength\extrarowheight{2pt}

\title{\bf Classificazione di cellule epiteliali HEp-2 mediante l'utilizzo dei tensori di Fisher} 
\date {27 Agosto 2015}
\author{Lorenzo Cioni\\\textit{{\small lore.cioni@gmail.com}}}

\begin{document}
\maketitle

\begin{abstract}

Analizzare e classificare le cellule epiteliali di tipo 2 (HEp-2) mediante l'utilizzo della tecnica della immunofluorescenza indiretta è uno standard per rilevare malattie al tessuto connettivo umano, come ad esempio l'Artrite Reumatoide. Purtroppo questo metodo è molto costoso in termini di tempo e di lavoro impiegato e particolarmente soggettivo.

Questo elaborato ha come finalità quella di implementare un metodo per la classificazione di questo tipo di cellule basato sull'utilizzo del descrittore di covarianza e dei tensori di Fisher per l'estrazione di \emph{features} dalle immagini.
\end{abstract}

\tableofcontents

\section{Introduzione}

Una delle procedure standard per il rilevamento di malattie al tessuto connettivo umano, come ad esempio l'Artrite Reumatoide o il Lupus, è l'utilizzo di Immunofluorescenza Indiretta sulle cellule epiteliali di tipo 2, altrimenti conosciute come HEp-2.

Questo tipo di analisi ha due principali svantaggi: è molto soggettiva e richiede un gran numero di ore lavorative. Si è così pensato ad un metodo per automatizzare il processo per ottenere risultati migliori sia sotto il profilo medico che dal punto di vista di tempo impiegato.

Il metodo proposto e implementato è tratto da un articolo pubblicato in occasione del contest di \emph{Pattern Recognition 2014 \footnote{ICPR Contest 2014 - \href{http://nerone.diem.unisa.it/hep2contest/description.shtml}{HEp-2 Cells Classification Contest}}} [1]. Per la classificazione delle cellule si procede inizialmente all'estrazione di un adeguato numero di \emph{features} attraverso l'utilizzo del \emph{Descrittore di Covarianza} [2], vengono poi utilizzati i \emph{Tensori di Fisher} che codificano informazioni addizionali rispetto alla distribuzione delle \emph{features} ed infine le cellule vengono classificate tramite un SVM multiclasse.

I test per la valutazione della bontà del metodo sono stati effettuati sul dataset della competizione\footnote{\href{http://mivia.unisa.it/datasets/biomedical-image-datasets/hep2-image-dataset/}{HEp-2 Dataset}}.

\section{Teoria}

\subsection{Covariance Descriptor}

\subsection{Fisher Tensors}
\section{Dataset}

Per la valutazione dell'efficienza dell'algoritmo implementato è stato utilizzato una parte del dataset dell'\emph{ICPR HEp-2 Cell Classification Contest}.

Il dataset contiene al suo interno 149 immagini che rappresentano le scansioni di vetrini di laboratorio. All'interno di queste immagini è possibile distinguere 9 diversi pattern: \emph{punteggiato}, \emph{nucleolare}, \emph{citoplasmico}, \emph{omogeneo}, \emph{granulare}, \emph{negativo} e \emph{centromero}. Alcune immagini del dataset sono poi classificate come \emph{altro}, rendendo difficile una loro classificazione, questi elementi sono stati dunque non considerati. A causa del basso numero di elementi classificati con \emph{centromero} anche questi sono stati rimossi fino ad ottenere un totale di 137 immagini.

Le immagini sono state acquisite tramite un microscopio a fluorescenza (40 ingrandimenti) in combinazione con una lampada di mercurio vaporizzato a 50W e una fotocamera digitale SLIM con risoluzione 1920 x 1110.

\begin{figure}[H] 
  \centering
    \includegraphics[width=0.9\textwidth]{images/example.jpg}
    \caption{{\small \textit{Esempio di immagine del dataset}}}
\end{figure}

Ciascuna immagine è stata poi annotata da un medico specialista ed associata ad una delle classi di pattern sopra esposte. Le annotazioni sono state inserite in una tabella così da poterne ricavare un \emph{training set}.

\begin{figure}[H]
\captionsetup[subfigure]{labelformat=empty}
\begin{subfigure}{.16\textwidth}
\centering
\includegraphics[height=1.6cm]{images/omogeneo.jpg}
\caption{Omogeneo}
\end{subfigure}%
\begin{subfigure}{.16\textwidth}
\centering
\includegraphics[height=1.8cm]{images/nucleolare.jpg}
\caption{Nucleolare}
\end{subfigure}%
\begin{subfigure}{.16\textwidth}
\centering
\includegraphics[height=1.8cm]{images/punteggiato.jpg}
\caption{Punteggiato}
\end{subfigure}%
\begin{subfigure}{.16\textwidth}
\centering
\includegraphics[height=1.8cm]{images/granulare.jpg}
\caption{Granulare}
\end{subfigure}%
\begin{subfigure}{.16\textwidth}
\centering
\includegraphics[height=1.8cm]{images/citoplasmico.jpg}
\caption{Citoplasmico}
\end{subfigure}%
\begin{subfigure}{.16\textwidth}
\centering
\includegraphics[height=1.8cm]{images/negativo.jpg}
\caption{Negativo}
\end{subfigure}%
\caption{Esempi di cellule classificate nel dataset}
\end{figure}

\section{Risultati}

In questa sezione vengono illustrati e documentati i risultati ottenuti mediante questo procedimento. 

I test sono stati effettuati considerando l'intero dataset di immagini e utilizzando la tecnica della classificazione con \emph{cross-validazione}. La cross-validazione (\emph{cross-validation} in inglese) è una tecnica statistica che consiste nel suddividere in $k$ parti equinumerose il dataset, \emph{$k$-fold cross-validation}, e, ad ogni passo, la parte $\frac{1}{k}$-esima del dataset viene ad essere il validation dataset, mentre la restante parte costituisce il training dataset. 

Così, per ognuna delle k parti (in questo caso $k = 8$) si allena il modello, evitando quindi problemi di \emph{overfitting}, ma anche di campionamento asimmetrico del training dataset, tipico della suddivisione del dataset in due sole parti (ovvero training e validation dataset). In altre parole, si suddivide il campione osservato in gruppi di egual numerosità, si esclude iterativamente un gruppo alla volta e lo si cerca di predire con i gruppi non esclusi. Ciò al fine di verificare la bontà del modello di predizione utilizzato.

Una volta costruito il modello tramite questa tecnica, ne viene valutata l'efficacia: tutto il dataset viene valutato e si ottengono così i risultati di corretta o errata classificazione.

Il valore di $K$ per la creazione del modello a mistura di gaussiane scelto è 16, valore con cui è stato ottenuto il punteggio più alto di accuratezza finale.

Nella Tabella 1 vengono raccolti i dati ottenuti dalla valutazione del dataset tramite il modello.

\begin{table}[H]
\centering
\footnotesize
\begin{tabular}{|l | c | c | c |} 
 \hline 
 \textbf{Classe} &  \textbf{Totale} & \textbf{Corretti} & \textbf{Percentuale} \\ [0.5ex] 
 \hline\hline
 Punteggiato & 49 & 38 & 77.55\\
 Nucleolare & 11 & 1 & 9.09\\
 Citoplasmico & 15 & 10 & 66.67\\
 Negativo & 33 & 30 & 90.91\\
 Omogeneo & 20 & 12 & 60.00\\
 Granulare & 9 & 2 & 22.22\\
 \hline
\end{tabular}
\caption{Risultati}
\label{table:1}
\end{table}

\begin{figure}[H] 
  \centering
    \includegraphics[width=0.9\textwidth]{images/confusion_matrix.png}
    \caption{{\small \textit{Matrice di confusione}}}
\end{figure}

Il miglior risultato ottenuto ha classificato correttamente 93 scansioni su 137, con un accuratezza totale del 67.88 \%.

Da una prima analisi sui risultati ottenuti si evince che il pattern \emph{punteggiato} è quello che ha migliore probabilità di essere correttamente classificato. Questo può dipendere anche dal fatto che è la classe con il maggior numero di esemplari e dunque il modello costruito è tendenzialmente più preciso.

Un ottimo risultato si ottiene anche per quanto riguarda il pattern \emph{negativo} che corrisponde a quelle cellule dove il marcatore fluorescente non emette luce visibile. 

I pattern \emph{nucleolare} e \emph{granulare} sono quelli che hanno ottenuto risultati peggiori. Il \emph{nucleolare}, ad esempio, viene riconosciuto correttamente una volta soltanto, confondendolo
in buona parte con il \emph{punteggiato} o con l'\emph{omogeneo}, mentre il pattern \emph{granulare} viene confuso quasi in modo equo fra tutte le altre classi. Questo può in parte dipendere da un minor numero di campioni su cui creare il modello ma anche da una varietà di cellule all'interno della stessa classe.
\section{Implementazione}

L'implementazione dell'algoritmo qui presentato è stata sviluppata interamente in codice Matlab, compatibile con la versione 2014a o successive. 
Per una corretta esecuzione è necessario avere installato alcuni pacchetti aggiuntivi:

\begin{itemize}
\item \emph{Image Acquisition Toolbox}: nesessario per l'elaborazione di immagini.
\item \emph{Image Processing Toolbox}: necessario per l'elaborazione di immagini.
\item \emph{Statistics and Machine Learning Toolbox}: necessario per la creazione del modello per la classificazione SVM e per la creazione di un modello a mistura di gaussiane.
\end{itemize}

Per la creazione del \emph{train set} di partenza viene utilizzata una funzione Matlab per il \emph{parsing} di valori da file in formato virgola mobile. In caso di diverse necessità sarà sufficiente modificare questa funzione senza interferire con il resto dell'esecuzione.

\subsection{Configurazioni iniziali}

Le configurazioni sono tutte racchiuse all'interno di un'unica classe, così da poter essere facilmente adattabile all'esecuzione su macchine diverse. 

Il file \textbf{configuration.m} contiene al suo interno le configurazioni di base del progetto: è necessario modificarle prima di proseguire con l'esecuzione del programma.

\begin{itemize}
\item \textbf{image\_path}: cartella dove si trovano le immagini da analizzare.
\item \textbf{image\_prefix}: prefisso del nome dei file.
\item \textbf{image\_ext}: estensione dei file.
\item \textbf{validation\_file}: file in formato CSV per la creazione del \emph{training set}.
\item \textbf{resize}: \emph{true/false}, se vero le immagini verranno ridimensionate. Permette un'esecuzione più rapida del programma, ma viene meno l'accuratezza finale.
\item \textbf{resizeTo}: imposta la dimensione alla quale ridimensionare l'immagine (se impostato il \emph{resize} a \emph{true}).
\item \textbf{K}: numero di gaussiane per la creazione del modello a mistura.
\end{itemize}

\subsection{Esecuzione}

L'algoritmo, come descritto nella Sezione 3 è stato suddiviso in 5 fasi fondamentali, da eseguire in ordine.

\begin{enumerate}
\item \textbf{loadTrainingSet.m}: si occupa della creazione del \emph{training set}. Legge il file CSV e va a creare una struttura contenente:
\begin{itemize}
\item ID dell'immagine.
\item \emph{Label} assegnata.
\item Nome completo del file e cartella.
\end{itemize}
\item \textbf{extractImages.m}: esegue quanto descritto nella Sezione 3.1. 
\item \textbf{runGMM.m}: esegue quanto descritto nella Sezione 3.2.
\item \textbf{saveSignatures.m}: calcola i tensori di Fisher nel modo descritto nella Sezione 3.3.
\item \textbf{runSVM.m}: esegue la classificazione. Viene stampata la tabella dei risultati ottenuti e mostrata una versione grafica della matrice di confusione (generata tramite la funzione \textbf{plotConfusionMatrix.m}).
\end{enumerate}

\section{Conclusioni}

La tecnica utilizzata per la classificazione di cellule epiteliali HEp-2 ha dato risultati positivi sul dataset considerato.

Buona parte, circa il 70\%, delle scansioni sono state correttamente classificate, in particolare la tecnica è molto buona per scartare i campioni negativi, riconosciuti con un'alta accuratezza.

Alcuni dei pattern sono però difficilmente classificabili: questo a causa anche del basso numero di campioni di training a disposizione che non consentono così la creazione di un modello adeguato. 

Un possibile miglioramento potrebbe essere ottenibile aumentando il numero di filtri di Gabor utilizzati, andando a estrarre un numero maggiore di osservazioni dall'immagine e rendendo più accurata la descrizione delle singole classi, andando però ad aumentare la durata di computazione.



In conclusione il metodo presentato in [1] e qui implementato è una buona soluzione al problema di classificazione di cellule, migliorabile nella fase di estrazione delle \emph{features} e nella scelta dei parametri.

\begin{thebibliography}{1}

\bibitem{}
Masoud Faraki, Mehrtash T. Harandi, Arnold Wiliem, Brian C. Lovell, \emph{Fisher tensors for classifying human epithelial cells}.\hskip 1em plus
  0.5em minus 0.4em\relax Pattern Recognition, Volume 47, 2014, pp. 2348 - 2359.

\bibitem{}
Oncel Tuzel, Fatih Porikli, Peter Meer, \emph{Region Covariance: A Fast Descriptor for
Detection and Classification}.\hskip 1em plus
  0.5em minus 0.4em\relax Mitsubishi Electric Research Laboratories, Inc., 2006.

\bibitem{}
Gabor, D.: \emph{Theory of communication} In J. IEE, vol. 93, pp. 429-457, London, 1946.

\end{thebibliography}



\end{document}