\section{Introduzione}

Una delle procedure standard per il rilevamento di malattie al tessuto connettivo umano, come ad esempio l'Artrite Reumatoide o il Lupus, è l'utilizzo di Immunofluorescenza Indiretta sulle cellule epiteliali di tipo 2, altrimenti conosciute come HEp-2.

Questo tipo di analisi ha due principali svantaggi: è molto soggettiva e richiede un gran numero di ore lavorative. Si è così pensato ad un metodo per automatizzare il processo per ottenere risultati migliori sia sotto il profilo medico che dal punto di vista di tempo impiegato.

Il metodo proposto e implementato è tratto da un articolo pubblicato in occasione del contest di \emph{Pattern Recognition 2014 \footnote{ICPR Contest 2014 - \href{http://nerone.diem.unisa.it/hep2contest/description.shtml}{HEp-2 Cells Classification Contest}}} [1]. Per la classificazione delle cellule si procede inizialmente all'estrazione di un adeguato numero di \emph{features} attraverso l'utilizzo del \emph{Descrittore di Covarianza} [2], vengono poi utilizzati i \emph{Tensori di Fisher} che codificano informazioni addizionali rispetto alla distribuzione delle \emph{features} ed infine le cellule vengono classificate tramite un SVM multiclasse.

I test per la valutazione della bontà del metodo sono stati effettuati sul dataset della competizione\footnote{\href{http://mivia.unisa.it/datasets/biomedical-image-datasets/hep2-image-dataset/}{HEp-2 Dataset}}.
