\section{Teoria}

Vengono ora presentate le tecniche utilizzate all'interno del programma: l'estrazione di \emph{features} dall'immagine mediante l'utilizzo dei \emph{filtri di Gabor}, il Descrittore di Covarianza (\emph{Covariance Descriptor}), il Modello Mistura di Gaussiane (\emph{Gaussian Mixture Model, GMM}) e i Tensori di Fisher (\emph{Fisher Tensors}).

\subsection{Il filtro di Gabor}

I filtri di Gabor\footnote{Gabor, D.: \emph{Theory of communication.} In J. IEE, vol. 93, pp. 429-457, Londra, 1946.} sono filtri passa-banda usati nell'analisi di immagini principalmente per l'estrazione di \emph{features} e l'analisi basata sulla tessitura.

La risposta finita all'impulso di questi filtri è calcolata come prodotto di uno sviluppo Gaussiano con oscillazione complessa. Estendendo queste funzioni a due dimensioni è possibile creare filtri sensibili all'orientazione\footnote{Daugman, J. G.: \emph{Uncertainty relation for resolution in space, spatial frequency, and orientation optimized by two-dimensional visual cortical filters} J. Optical Society of America A, vol. 2, no. 7, pp. 1160-1169, July 1985.} e sotto certe condizioni è possibile approssimare linearmente la fase.

Sia $(x, y)$ un punto dell'immagine. L'equazione per il filtro di Gabor 2D è la seguente:

$$G(x, y) = e^{(- \frac{(x')^2 + \gamma^2 (y')^2}{2 \sigma^2})} \cos (2 \pi \frac{x'}{\lambda})  $$ 

con 

$$x' = x \cos \theta + y \sin \theta$$
$$y' = - x \cos \theta + y \sin \theta$$



\subsection{Covariance Descriptor}

\subsection{Fisher Tensors}